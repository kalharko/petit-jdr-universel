% !TEX TS-program = pdflatex
% !TEX encoding = UTF-8 Unicode

% Article template :)

% General stuff
\documentclass[11pt]{article} % use larger type; default would be 10pt

\usepackage[utf8]{inputenc} % set input encoding (not needed with XeLaTeX)
\usepackage[french]{babel} % Language setting
\usepackage [T1] {fontenc}
\pdfmapfile{Overpass.map}

% Geometry
\usepackage{geometry}
\geometry{a4paper}
\geometry{margin=2cm}

% Usefull packages
\usepackage{eso-pic} % to add background images to the article
\usepackage{graphicx} % support the \includegraphics command and options
%\usepackage[colorlinks=false]{hyperref}
\usepackage{titlesec} % to change default styles of /sections, /title ...
\usepackage{xcolor}
\usepackage{tabto}
%\usepackage{amssymb}
\usepackage{amssymb}
\usepackage{multirow}
\usepackage{tikz}
\usetikzlibrary{arrows,automata,positioning}
\usepackage{amsmath}
\usepackage{lettrine}
\usepackage{oldgerm}
\usepackage{yfonts}
\usepackage{multicol}
\usepackage{blindtext}
\usepackage{svg}
\usepackage{tabularx}
\usepackage{hyperref}
\hypersetup{
    colorlinks=true,
    linkcolor=blue,
    filecolor=magenta,
    urlcolor=cyan,
    pdftitle={Overleaf Example},
    pdfpagemode=FullScreen,
    }

\newcommand{\enluminure}[2]{\lettrine[lines=3]{\small \initfamily #1}{#2}}


% Style !
\definecolor{10Black}{cmyk}{0, 0, 0, 0.10} % 10% of black
\definecolor{20Black}{cmyk}{0, 0, 0, 0.20} % 20% of black
\definecolor{80Black}{cmyk}{0, 0, 0, 0.80} % 80% of black
\definecolor{95Black}{cmyk}{0, 0, 0, 0.95} % 60% of black
%\color{10Black}
%\pagecolor{black}

% New commands
\newcommand{\myjump}[1][1]{\mbox{}\\[#1cm]}


\begin{document}
\pagestyle{empty}

\begin{center}
    \textbf{Coups Spéciaux}

    Règles additionnelles
\end{center}


\myjump
TODO

\begin{itemize}
    \item comment créer un coup spécial
    \item coûte 2 points de compétence
    \item coup spécial : pas 2 fois sur le même groupe d'ennemi
    \item
    \item
    \item combien dé nnemi ca touche
    \item est-ce-que stress +1
    \item effet sur cible
    \item
    \item effet negatif sur utilisateur
    \item
    \item attaque
    \item attaque dans le dos = +1d6
    \item 2 cibles = degats complet et degats /2 1fois
    \item 3 ou plus cibles = degats/2 1fois
    \item
    \item defense = +3 def pendant un tour 1fois
    \item fuite = donne du déplacement 1fois
    \item
    \item stun 1 fois
    \item charge = degats/2 1 fois
    \item pousser 1 fois
    \item augmenter oc cible
    \item cracheur de feu = attaque normale
    \item
    \item provocation n'est pas un coup spécial

\end{itemize}


\end{document}
