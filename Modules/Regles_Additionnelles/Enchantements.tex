% !TEX TS-program = pdflatex
% !TEX encoding = UTF-8 Unicode

% Article template :)

% General stuff
\documentclass[11pt]{article} % use larger type; default would be 10pt

\usepackage[utf8]{inputenc} % set input encoding (not needed with XeLaTeX)
\usepackage[french]{babel} % Language setting
\usepackage [T1] {fontenc}
\pdfmapfile{Overpass.map}

% Geometry
\usepackage{geometry}
\geometry{a4paper}
\geometry{margin=1.5cm}

% Usefull packages
\usepackage{eso-pic} % to add background images to the article
\usepackage{graphicx} % support the \includegraphics command and options
%\usepackage[colorlinks=false]{hyperref}
\usepackage{titlesec} % to change default styles of /sections, /title ...
\usepackage{xcolor}
\usepackage{tabto}
%\usepackage{amssymb}
\usepackage{amssymb}
\usepackage{multirow}
\usepackage{tikz}
\usetikzlibrary{arrows,automata,positioning}
\usepackage{amsmath}
\usepackage{lettrine}
\usepackage{oldgerm}
\usepackage{yfonts}
\usepackage{multicol}
\usepackage{blindtext}
\usepackage{svg}
\usepackage{tabularx}
\usepackage{hyperref}
\hypersetup{
    colorlinks=true,
    linkcolor=blue,
    filecolor=magenta,
    urlcolor=cyan,
    pdftitle={Overleaf Example},
    pdfpagemode=FullScreen,
    }

\newcommand{\enluminure}[2]{\lettrine[lines=3]{\small \initfamily #1}{#2}}
\newcommand{\scbf}[1]{\textsc{\textbf{#1}}}

% Style !
\definecolor{10Black}{cmyk}{0, 0, 0, 0.10} % 10% of black
\definecolor{20Black}{cmyk}{0, 0, 0, 0.20} % 20% of black
\definecolor{80Black}{cmyk}{0, 0, 0, 0.80} % 80% of black
\definecolor{95Black}{cmyk}{0, 0, 0, 0.95} % 60% of black
%\color{10Black}
%\pagecolor{black}

% New commands
\newcommand{\myjump}[1][1]{\mbox{}\\[#1cm]}


\begin{document}
\pagestyle{empty}

\begin{center}
    \textbf{Enchantements}

    Module de règles optionelles
\end{center}



\enluminure{E}{nchanter} un équipement permet à un joueur d'améliorer les capacités de son personnage grâce à sa créativité. Une fois que le MJ a écouté la description du sort d'un joueur, il doit faire un travail d'interprétation pour en déterminer son coût et ses effets de gameplay.




\myjump[0]
\textbf{\huge Description et coûts}\newline
La première étape pour créer un enchantement est d'écrire ou demander au joueur sa description puis déterminer son coût. Avant de parcourir les tableaux suivant pour déterminer le coût, le sort commence avec les caractéristiques suivantes:
Charges max 1, durée 1 tour de combat, porté contacte

\myjump[0]
\begin{tabularx}{\linewidth}{|Xc|}
\hline

    \textbf{Effet} & \textbf{Coût}\\
    \hline
    \textbf{Généralitées} &\\
    x effets différents & +x\\
    Coût arbitraire du MJ & +x\\
    Charges max + 1 & +1\\
    Manipule une statistique de la feuille de perso & +1\\
    Objet fragile qui doit être rendu incassable & +1\\
    Inflige des dégâts magiques & +1\\
    \hline
    \textbf{Durée} &\\
    Tour de combat +1 & +1\\
    Jusqu'à la fin du combat & +3\\
    Effet passif & +4\\
    \hline
    \textbf{Cibles} &\\
    Portée + 10m & +1\\
    Portée à vue & +4\\
    Diamètre zone +1 & +2\\
    Nombre cibles max dans zone +1 & +1\\
    \hline
    \textbf{Invocation} &\\
    Créer ou invoquer une entitée & +1\\
    +1D6 au nombre d'entitées invoquées & +2\\
    Taille de l'invocation & +1\\
    Permet de contrôler l'entité invoquée & +1\\
    \hline
    \textbf{Limitations} &\\
    Ne s'applique que à un type de de cibles (morts-vivants, non magiques ...) & -1\\
    Ne foncionne que dans un biome spécifique & -1\\
    Ne peut pas toucher le boss & -1\\
    La cible peut esquiver le sort & -2\\
    Augmente le stress de l'utilisateur & -1\\
    Augmente le stress de tout le monde & -1\\
    Utilisable unique dans une range de PV & -1\\

\hline
\end{tabularx}




\myjump[0.35]
\begin{tabularx}{\linewidth}{|Xc|}
\hline

    \textbf{Puissance} & \textbf{Multiplicateur}\\
    \hline
    Pas d'impacte directe sur le gameplay & x0.5 arrondie à l'inférieur\\
    & consome une charge par jour\\
    Illusion & x0.5 arrondie au supérieur\\
    Effets reproductibles avec le bon equipement & x1\\
    Effets non reproductibles sans magie mais impacte modéré & x1.5 arrondie au supérieur\\
    Impacte fort & x2\\
    Manipule l'esprit de la cible & x3\\
    Méprise les loies naturelle, magie noire, mort & x5\\


\hline
\end{tabularx}

\newpage
\textbf{\huge Définition du rituel}\newline
Déterminer le coût d'un enchantement donne une estimation de sa puissance et permet de définir quels sont les efforts et coûts à payer pour pouvoir réaliser le rituel.

\myjump[0]
\textbf{Tableau de définition du rituel}\newline
\begin{tabularx}{\linewidth}{|c|X|}
\hline
    \textbf{Coût} & \textbf{Forme du rituel}\\
    \hline
    1    & Dans un lieu calme avec 3 ingrédients communs\\
    3 & + 1d6 jours d'études dans une bibliothèque\\
    5 & + 1 ingrédient rare\\
    x & + 1d6 jours d'études dans une bibliothèque\\
    x & + sacrifice\\
    15 & + 1 ingrédient sanglant\\
    x & + doit sacrifier un artefacte magique\\
    X & + doit se nourrir de la peine d'un être cher\\
    X & + doit se nourrir de la peine d'un village\\
    X & + doit se nourrir de la peine d'un pays\\

\hline
\end{tabularx}


\myjump[0.35]
\textbf{Ingrédient commun} \tabto{4.3cm}Facilement récoltable avec les bonnes connaissances. Ne coûte pas trop cher.\newline
\textbf{Ingrédient rare} \tabto{4.3cm}Difficilement récoltable, coûte suffisament cher pour rendre vos joueurs pauvres.\newline
\textbf{Ingrédient sanglant} \tabto{4.3cm}L'enchantement doit être forgé dans le coeur encore fumant d'un monstre.\newline
\textbf{Sacrifice} \tabto{4.3cm}Lancer 1d6 dans le tableau des sacrifices.\newline
\textbf{Nourir d'une peine} \tabto{4.3cm}Pour les sorts plus maléfiques, requiert de créer de la souffrance. Empoisonner le puit d'un village pour récolter sa peine par exemple.



\myjump[0]
\textbf{Tableau des Sacrifices}\newline\noindent
\begin{tabularx}{\linewidth}{|c|X|}
\hline
    \scbf{No} & \scbf{Description}\\
    \hline
    1 & Perd des points de compétences\\
    2 & Limite basse de stress est augmenté de 1\\
    3 & Sacrifier un membre\\
    4 & Amnésie (discrétion du MJ)\\
    5 & Perd des de PV max\\
    6 & Ignore stress 1\\\hline
\end{tabularx}


\myjump[0.35]
\textbf{\huge Équilibrage}\newline
Il peut arriver que en tant que MJ vous n'aimez pas l'enchantement qu'un de vos joueurs veut créer, mais que vous ne voulez pas brimer sa créativité. Dans ce cas, vous pouvez lui proposer d'ajouter des limitations à son sort.


\end{document}
