% !TEX TS-program = pdflatex
% !TEX encoding = UTF-8 Unicode

% Article template :)

% General stuff
\documentclass[11pt]{article} % use larger type; default would be 10pt

\usepackage[utf8]{inputenc} % set input encoding (not needed with XeLaTeX)
\usepackage[french]{babel} % Language setting
\usepackage [T1] {fontenc}
\pdfmapfile{Overpass.map}

% Geometry
\usepackage{geometry}
\geometry{a4paper}
\geometry{margin=1cm}

% Usefull packages
\usepackage{eso-pic} % to add background images to the article
\usepackage{graphicx} % support the \includegraphics command and options
%\usepackage[colorlinks=false]{hyperref}
\usepackage{titlesec} % to change default styles of /sections, /title ...
\usepackage{xcolor}
\usepackage{tabto}
%\usepackage{amssymb}
\usepackage{amssymb}
\usepackage{multirow}
\usepackage{tikz}
\usetikzlibrary{arrows,automata,positioning}
\usepackage{amsmath}
\usepackage{lettrine}
\usepackage{oldgerm}
\usepackage{yfonts}
\usepackage{multicol}
\usepackage{blindtext}
\usepackage{svg}
\usepackage{tabularx}
\usepackage{hyperref}
\hypersetup{
    colorlinks=true,
    linkcolor=blue,
    filecolor=magenta,
    urlcolor=cyan,
    pdftitle={Overleaf Example},
    pdfpagemode=FullScreen,
    }

\newcommand{\enluminure}[2]{\lettrine[lines=3]{\small \initfamily #1}{#2}}


% Style !
\definecolor{10Black}{cmyk}{0, 0, 0, 0.10} % 10% of black
\definecolor{20Black}{cmyk}{0, 0, 0, 0.20} % 20% of black
\definecolor{80Black}{cmyk}{0, 0, 0, 0.80} % 80% of black
\definecolor{95Black}{cmyk}{0, 0, 0, 0.95} % 60% of black
%\color{10Black}
%\pagecolor{black}

% New commands
\newcommand{\myjump}[1][1]{\mbox{}\\[#1cm]}


\begin{document}
\pagestyle{empty}

\begin{center}
    \textbf{Fiche Donjon}

    Outils
\end{center}



\begin{tabularx}{\linewidth}{X}
\hline
    \textsc{\textbf{Nom/Titre}} \\\hline
    \textsc{\textbf{Description}} \vspace{5cm} \\\hline

    \textsc{\textbf{Plan}} \vspace{20cm} \\

\hline
\end{tabularx}






\newpage\noindent
\begin{tabularx}{\linewidth}{c|X}
\hline
    \textsc{\textbf{Nom/Titre}} & \\\hline
    \multicolumn{2}{@{}l}{
    \begin{tabularx}{\linewidth}{cXcXcX}
        \textsc{\textbf{Degats haut}} & & \textsc{\textbf{Degats bas}} & & \textsc{\textbf{PV}} & \\\hline
    \end{tabularx}
    }\\
    Description et & \\
    Coups spéciaux & \\
    \\\\\\
    \textsc{\textbf{1}} & \\
\hline

    \multicolumn{2}{l}{\vspace{0.25cm}} \\
\hline
    \textsc{\textbf{Nom/Titre}} & \\\hline
    \multicolumn{2}{@{}l}{
    \begin{tabularx}{\linewidth}{cXcXcX}
        \textsc{\textbf{Degats haut}} & & \textsc{\textbf{Degats bas}} & & \textsc{\textbf{PV}} & \\\hline
    \end{tabularx}
    }\\
    Description et & \\
    Coups spéciaux & \\
    \\\\\\
    \textsc{\textbf{2}} & \\
\hline

    \multicolumn{2}{l}{\vspace{0.25cm}} \\
\hline
    \textsc{\textbf{Nom/Titre}} & \\\hline
    \multicolumn{2}{@{}l}{
    \begin{tabularx}{\linewidth}{cXcXcX}
        \textsc{\textbf{Degats haut}} & & \textsc{\textbf{Degats bas}} & & \textsc{\textbf{PV}} & \\\hline
    \end{tabularx}
    }\\
    Description et & \\
    Coups spéciaux & \\
    \\\\\\
    \textsc{\textbf{3}} & \\
\hline

    \multicolumn{2}{l}{\vspace{0.25cm}} \\
\hline
    \textsc{\textbf{Nom/Titre}} & \\\hline
    \multicolumn{2}{@{}l}{
    \begin{tabularx}{\linewidth}{cXcXcX}
        \textsc{\textbf{Degats haut}} & & \textsc{\textbf{Degats bas}} & & \textsc{\textbf{PV}} & \\\hline
    \end{tabularx}
    }\\
    Description et & \\
    Coups spéciaux & \\
    \\\\\\
    \textsc{\textbf{4}} & \\
\hline


    \multicolumn{2}{l}{\vspace{0.25cm}} \\
\hline
    \textsc{\textbf{Nom/Titre}} & \\\hline
    \multicolumn{2}{@{}l}{
    \begin{tabularx}{\linewidth}{cXcXcX}
        \textsc{\textbf{Degats haut}} & & \textsc{\textbf{Degats bas}} & & \textsc{\textbf{PV}} & \\\hline
    \end{tabularx}
    }\\
    Description et & \\
    Coups spéciaux & \\
    \\\\\\
    \textsc{\textbf{5}} & \\
\hline

    \multicolumn{2}{l}{\vspace{0.25cm}} \\
\hline
    \textsc{\textbf{Loot}} & \\\cline{1-1}
    \multicolumn{2}{l}{
    \begin{tabularx}{\linewidth}{X}
        \\\\\hline
        \\\\\hline
        \\\\\hline
        \\\\\hline
        \\\\\hline
        \\\\\hline
    \end{tabularx}
    }\\

%\hline
\end{tabularx}


\end{document}
