% !TEX TS-program = pdflatex
% !TEX encoding = UTF-8 Unicode

% Article template :)

% General stuff
\documentclass[11pt]{article} % use larger type; default would be 10pt

\usepackage[utf8]{inputenc} % set input encoding (not needed with XeLaTeX)
\usepackage[french]{babel} % Language setting
\usepackage [T1] {fontenc}
\pdfmapfile{Overpass.map}

% Geometry
\usepackage{geometry}
\geometry{a4paper}
\geometry{margin=2cm}

% Usefull packages
\usepackage{eso-pic} % to add background images to the article
\usepackage{graphicx} % support the \includegraphics command and options
%\usepackage[colorlinks=false]{hyperref}
\usepackage{titlesec} % to change default styles of /sections, /title ...
\usepackage{xcolor}
\usepackage{tabto}
%\usepackage{amssymb}
\usepackage{amssymb}
\usepackage{multirow}
\usepackage{tikz}
\usetikzlibrary{arrows,automata,positioning}
\usepackage{amsmath}
\usepackage{lettrine}
\usepackage{oldgerm}
\usepackage{yfonts}
\usepackage{multicol}
\usepackage{blindtext}
\usepackage{svg}
\usepackage{tabularx}
\usepackage{hyperref}
\hypersetup{
    colorlinks=true,
    linkcolor=blue,
    filecolor=magenta,
    urlcolor=cyan,
    pdftitle={Overleaf Example},
    pdfpagemode=FullScreen,
    }

%-----------------------
% Newcommands
%-----------------------
\newcommand{\enluminure}[2]{\lettrine[lines=3]{\small \initfamily #1}{#2}}
\newcommand{\scbf}[1]{\textsc{\textbf{#1}}}

\newcounter{NoCounter}
\setcounter{NoCounter}{0}
\newcommand{\myline}[3]{
    \stepcounter{NoCounter}
    \arabic{NoCounter} & #1 & #2 & #3\\
}


% Style !
\definecolor{10Black}{cmyk}{0, 0, 0, 0.10} % 10% of black
\definecolor{20Black}{cmyk}{0, 0, 0, 0.20} % 20% of black
\definecolor{80Black}{cmyk}{0, 0, 0, 0.80} % 80% of black
\definecolor{95Black}{cmyk}{0, 0, 0, 0.95} % 60% of black
%\color{10Black}
%\pagecolor{black}

% New commands
\newcommand{\myjump}[1][1]{\mbox{}\\[#1cm]}


\begin{document}
\pagestyle{empty}

\begin{center}
    \textbf{L'Alchimie à Jamy}

    Module de règles optionelles
\end{center}

\enluminure{L}{'alchimie} pour les nuls


\myjump
Lorsqu'un personnage alchimiste rencontre un nouvel ingrédient ou élément, il peut le tester sur lui-même ou un coéquipié pour en déterminer son effet. Lancer 1d20 dans la table suivante :\newline

\myjump[0]
\begin{tabularx}{\linewidth}{|c|c|c|X|}
\hline

    \scbf{No} & \scbf{Effet} & \scbf{Niveau X} & \scbf{Description}\\\hline

    \myline{Paralysie}{1d6}{Paralyse pendant X tours de combat}

    \myline{Ignore armure}{1d6}{Inflige X dǵats ignore armure}

    \myline{Soin}{2d6}{Soigne de X}

    \myline{Ursidé}{1d6}{Transforme en ours}

\hline
\end{tabularx}



\newpage
a



%-----------------
% Annexe
%-----------------
\newpage
\newgeometry{margin=1cm}
\begin{center}
    \textbf{L'alchimie à Jamy}

    Annexe pour les joueurs
\end{center}

\noindent
\begin{tabularx}{\linewidth}{p{4cm}c|cX}
\hline

    \scbf{Nom} &&& \scbf{Description}\\\hline

    &&&\\ &&&\\\cline{1-1}\cline{4-4}
    &&&\\ &&&\\\cline{4-4}
    &&&\\ &&&\\\cline{1-1}\cline{4-4}
    &&&\\ &&&\\\cline{4-4}
    &&&\\ &&&\\\cline{1-1}\cline{4-4}
    &&&\\ &&&\\\cline{4-4}
    &&&\\ &&&\\\cline{1-1}\cline{4-4}
    &&&\\ &&&\\\cline{4-4}
    &&&\\ &&&\\\cline{1-1}\cline{4-4}
    &&&\\ &&&\\\cline{4-4}
    &&&\\ &&&\\\cline{1-1}\cline{4-4}
    &&&\\ &&&\\\cline{4-4}
    &&&\\ &&&\\\cline{1-1}\cline{4-4}
    &&&\\ &&&\\\cline{4-4}
    &&&\\ &&&\\\cline{1-1}\cline{4-4}
    &&&\\ &&&\\\cline{4-4}
    &&&\\ &&&\\\cline{1-1}\cline{4-4}
    &&&\\ &&&\\\cline{4-4}
    &&&\\ &&&\\\cline{1-1}\cline{4-4}
    &&&\\ &&&\\\cline{4-4}
    &&&\\ &&&\\\cline{1-1}\cline{4-4}
    &&&\\ &&&\\\cline{4-4}
    &&&\\ &&&\\\cline{1-1}\cline{4-4}
    &&&\\ &&&\\\cline{4-4}
    &&&\\ &&&\\\cline{1-1}\cline{4-4}
    &&&\\ &&&\\\cline{4-4}
    &&&\\ &&&\\\cline{1-1}\cline{4-4}

\end{tabularx}

\newpage
\noindent
\begin{tabularx}{\linewidth}{p{4cm}c|cX}
\hline

    \scbf{Nom} &&& \scbf{Effet}\\\hline

    &&&\\ &&&\\\cline{1-1}\cline{4-4}
    &&&\\ &&&\\\cline{4-4}
    &&&\\ &&&\\\cline{1-1}\cline{4-4}
    &&&\\ &&&\\\cline{4-4}
    &&&\\ &&&\\\cline{1-1}\cline{4-4}
    &&&\\ &&&\\\cline{4-4}
    &&&\\ &&&\\\cline{1-1}\cline{4-4}
    &&&\\ &&&\\\cline{4-4}
    &&&\\ &&&\\\cline{1-1}\cline{4-4}
    &&&\\ &&&\\\cline{4-4}
    &&&\\ &&&\\\cline{1-1}\cline{4-4}
    &&&\\ &&&\\\cline{4-4}
    &&&\\ &&&\\\cline{1-1}\cline{4-4}
    &&&\\ &&&\\\cline{4-4}
    &&&\\ &&&\\\cline{1-1}\cline{4-4}
    &&&\\ &&&\\\cline{4-4}
    &&&\\ &&&\\\cline{1-1}\cline{4-4}
    &&&\\ &&&\\\cline{4-4}
    &&&\\ &&&\\\cline{1-1}\cline{4-4}
    &&&\\ &&&\\\cline{4-4}
    &&&\\ &&&\\\cline{1-1}\cline{4-4}
    &&&\\ &&&\\\cline{4-4}
    &&&\\ &&&\\\cline{1-1}\cline{4-4}
    &&&\\ &&&\\\cline{4-4}
    &&&\\ &&&\\\cline{1-1}\cline{4-4}
    &&&\\ &&&\\\cline{4-4}
    &&&\\ &&&\\\cline{1-1}\cline{4-4}
    &&&\\ &&&\\\cline{4-4}

\end{tabularx}


\end{document}
