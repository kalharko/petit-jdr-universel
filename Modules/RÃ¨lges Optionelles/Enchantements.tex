% !TEX TS-program = pdflatex
% !TEX encoding = UTF-8 Unicode

% Article template :)

% General stuff
\documentclass[11pt]{article} % use larger type; default would be 10pt

\usepackage[utf8]{inputenc} % set input encoding (not needed with XeLaTeX)
\usepackage[french]{babel} % Language setting
\usepackage [T1] {fontenc}
\pdfmapfile{Overpass.map}

% Geometry
\usepackage{geometry}
\geometry{a4paper}
\geometry{margin=2cm}

% Usefull packages
\usepackage{eso-pic} % to add background images to the article
\usepackage{graphicx} % support the \includegraphics command and options
%\usepackage[colorlinks=false]{hyperref}
\usepackage{titlesec} % to change default styles of /sections, /title ...
\usepackage{xcolor}
\usepackage{tabto}
%\usepackage{amssymb}
\usepackage{amssymb}
\usepackage{multirow}
\usepackage{tikz}
\usetikzlibrary{arrows,automata,positioning}
\usepackage{amsmath}
\usepackage{lettrine}
\usepackage{oldgerm}
\usepackage{yfonts}
\usepackage{multicol}
\usepackage{blindtext}
\usepackage{svg}
\usepackage{tabularx}
\usepackage{hyperref}
\hypersetup{
    colorlinks=true,
    linkcolor=blue,
    filecolor=magenta,
    urlcolor=cyan,
    pdftitle={Overleaf Example},
    pdfpagemode=FullScreen,
    }

\newcommand{\enluminure}[2]{\lettrine[lines=3]{\small \initfamily #1}{#2}}


% Style !
\definecolor{10Black}{cmyk}{0, 0, 0, 0.10} % 10% of black
\definecolor{20Black}{cmyk}{0, 0, 0, 0.20} % 20% of black
\definecolor{80Black}{cmyk}{0, 0, 0, 0.80} % 80% of black
\definecolor{95Black}{cmyk}{0, 0, 0, 0.95} % 60% of black
%\color{10Black}
%\pagecolor{black}

% New commands
\newcommand{\myjump}[1][1]{\mbox{}\\[#1cm]}


\begin{document}
\pagestyle{empty}

\begin{center}
    \textbf{Enchantements}

    Module de règles optionelles
\end{center}

\enluminure{C}{e} module permet la créations d'enchantements à partir d'une description donnée par un joueur ou le MJ. De plus de créer des armes intéressantes, réunir les ingrédients et les conditions nécessaire au rituel peut être un appel à l'aventure. Finalement, les armes que les joueurs enchanterons seront beaucoup plus personnelles et auront une histoire.

\myjump
\begin{itemize}
    \item Description du sort
    \item Magie de bataille ou élémentaliste ?
    \item Magie interdite ?
    \item 1d4 semaines de 8h de recherche par jours
    \item Calculer coût de base grâce au tableau
    \item Identifier le niveau du sort
    \item Coût en PM
    \item Composants associés au sort
    \item Seconde période de recherche qui dure une semaine par coût en PM
    \item 300 PE par niveau du sort
    \item


\end{itemize}


\myjump[0]
\textbf{\huge Description et coûts}\newline
La première étape pour créer un enchantement est de créer sa description et déterminer son coût. Parcourir les tableaux suivants pour trouver le coût finale du sort.\newline


\noindent
\begin{tabularx}{\linewidth}{|Xc|}
\hline

    \textbf{Effet} & \textbf{Coût}\\
    \hline
    \textbf{Généralitées} &\\
    Coût initiale & +1\\
    Durée du sort +1 & +1\\
    Effet passif & +3\\
    \hline
    \textbf{Cibles} &\\
    Contacte & +1\\
    Portée + 4 & +2\\
    Diamètre zone +1 & +2\\
    Nombre cibles max dans zone +1 & +1\\
    \hline
    \textbf{Invocation} &\\
    Créer ou invoquer une entitée & +1\\
    +1D6 au nombre d'entitées invoquées & +2\\
    Permet de contrôler l'entité invoquée & +1\\
    \hline
    \textbf{Limitations} &\\
    Ne s'applique que à un type de de cibles (morts-vivants, non magiques ...) & -2\\
    La cible peut esquiver le sort & -2\\


\hline
\end{tabularx}

\myjump[0.35]\noindent
\begin{tabularx}{\linewidth}{|Xc|}
\hline

    \textbf{Puissance} & \textbf{Multiplicateur}\\
    \hline
    Pas d'impacte directe sur le gameplay & x0.5 arrondie à l'inférieur\\
    Illusion & x0.5 arrondie au supérieur\\
    Effets reproductibles avec le bon equipement & x1\\
    Effets non reproductibles sans magie mais impacte modéré & x1.5 arrondie au supérieur\\
    Impacte fort & x2\\
    Manipule l'esprit de la cible & x3\\
    Méprise les loies naturelle, magie noire, mort & x5\\


\hline
\end{tabularx}


\newpage
\textbf{\huge Définition du rituel}\newline
Pour appliquer un enchantement à un objet, il faut prendre du temps d'étude, réunir des ingrédients et trouver le lieu parfait

\begin{tabularx}{\linewidth}{|cX|}
\hline

    \textbf{Coût} & \textbf{Forme du rituel}\\
    \hline
    x & Dans un lieu calme avec 3 ingrédients communs\\
    x & + 1d6 jours d'études dans une bibliothèque, 1 ingrédient rare\\
    x & + \\
    x & + 1d6 jours d'études dans une bibliothèque\\
    x & + 1 ingrédient sanglant\\


\hline
\end{tabularx}


\myjump
Tableau ingrédients communs + prix\newline
Tableau ingrédients rares + prix  \newline
Tableau ingrédients sanglants\newline




\end{document}
